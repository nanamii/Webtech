%\documentclass[ngerman,dvips]{wkcms}  % latex + dvips
\documentclass[ngerman,pdf]{wkcms}    % pdflatex

\usepackage[utf8]{inputenc}

\title{SSH - The Secure Shell}
\subtitle{Überblick, Technologie, Anwendungsbeispiele}
\author{Susanne Kießling}
\date{\todaylong}

\address
{Hochschule Augsburg\\
 Masterstudiengang Informatik\\
 E-Mail: \url{susanne.kiessling@hs-augsburg.de}      
}

\begin{abstract}
Dfsjkhc sdjfh sdjlkfh sdljkf hlasdjkf haljksdfh asdjlfh als. asdfjk asdjflkha
sdfljhas dfljashd fljkasdhf jlaksdfh lajskdfh alsjkdfh asldjkf hasldjkfh
asljkdfh asljkdf hasljdkfh asljkdfh asljkdfh asljkdfh asljkdfh asljkdfh
asljkdfh alsjkdfh asljkdfh asljkdfh asljkdfh asljkdfh asldkjfh asldkjfh
aslkdjfh.

Asljkdfh alsjkdfh asljkdfh asljkdfh asljkdfh asljkdfh asldkjfh asldkjfh
asljkdfh alsjkdfh asljkdfh asljkdfh asljkdfh asljkdfh asldkjfh asldkjfh
asljkdfh alsjkdfh asljkdfh asljkdfh asljkdfh asljkdfh asldkjfh asldkjfh
aslkdjfh.
\end{abstract}

\keywords{SSH, OpenSSH, Verschlüsselung, TCP/IP-Protokoll}
\categories{Allgemein}

\begin{document}

\maketitle


\section{Einleitung}

Die Übertragung von Daten innerhalb von Computernetzwerken geschieht ohne
entsprechende Vorkehrungen grundsätzlich ungesichert. E-Mails, Passwörter und sämtliche Daten, die übertragen werden, können abgefangen und gelesen werden. SSH (Secure Shell) ist ein Protokoll, das die verschlüsselte Übertragung von Daten über ein Netzwerk ermöglicht \cite{SSH}.
Trotz zahlreicher Berichterstattung über die Folgen von sorglosem Umgang mit Daten in Computernetzwerken, wozu natürlich auch das World Wide Web gehört, ändern viele Nutzer
ihr Verhalten diesbezüglich nicht. Die Sicherheit von privaten Daten ist ein hochaktuelles Thema. SSH (als Protokoll) und dessen Implementierungen stellen eine Möglichkeit dar, Daten innerhalb eines Computernetzwerks sicher zu übertragen. Ziel ist es nun, einen Überblick zu SSH zu geben. OpenSSH dient anschließend als Beispiel-Implementierung, anhand derer die Funktionalitäten gezeigt werden. Der Ablauf eines Verbindungsaufbaus und die verwendeten Verschlüsselungsalgorithmen werden erläutert. Auf aktuelle sicherheitskritische BUGs wird eingegangen. Zur Abgrenzung zwischen der Bezeichnung von SSH als Protokoll bzw. als Implementierung, wird an dieser Stelle festgelegt, dass \IT{SSH} das Protokoll bezeichnet und die Implementierungen explizit benannt werden (bsp. \IT{OpenSSH}).

--> Bereits 19.. entwickelt, in zahlreichen Produkten implementiert
--> Vorgänger telnet, rsh, alles ungesichert

--> evtl. als aktuell erwähnen, dass es auch SSH für mobile Betriebssysteme gibt

 schlechte Passwörter/Bruce Schneier?

SSH (Secure Shell) ist eine unixoide Tralalala...



\newpage

\section{Grundlegendes zu SSH}


SSH verschlüsselt die forlaufende Verbindung zwischen Computern in einem Netzwerk. Dazu verwendet SSH eine Client/Server-Architektur. Abbildung \ref{fig:abb1} veranschaulicht das Grundprinzip von SSH, das laut seiner Spezifikation die Authentifizierung, die Verschlüsselung und die Integrität von Daten, die in einem Netzwerk übertragen werden, beinhaltet.

\begin{figure}[!h] 
\centering 
\includegraphics[scale=0.25]{./images/grundprinzip.png}
\label{fig:abb1} 
\caption[Abbildung 1]{Grundprinzip von SSH, eigene Darstellung.} 
\end{figure}
 
Wie in der Abbildung \ref{fig:abb1} zu sehen ist, verlangt eine SSH-Verbindung zwei Authentifizierungen \cite[S. 47]{SSH}. Der Server authentifiziert sich beim Client und umgekehrt (näheres zum Ablauf unter \ref{sec:ablauf}). Zum Verschlüsseln der Daten werden zufällige Sitzungsschlüssel ausgehandelt. Nach Beenden der Sitzung verlieren diese Schlüssel ihre Gültigkeit. SSH garantiert die Integrität der übertragenen Daten, \Dh die Daten können nicht durch einen Dritten (Mallory) verändert werden. 


\subsection{Protokolle}

Das SSH-Protokoll existiert in zwei unterschiedlichen Versionen (SSH-1.x und SSH-2.x), die nicht kompatibel sind. SSH-1 wurde 1995 von Tatu Ylönen an der University of Technology in Helsinki entwickelt. Da es Mängel in der Integritätsprüfung aufweist, ist eine weitere Verwendung des SSH-1 Protokolls nicht empfehlenswert.

SSH-2 ist modular aufgebaut. Die Architektur besteht aus drei Hauptkomponenten: Die Transportschicht, die Authentifizierungsschicht und die Verbindungsschicht. Beschrieben sind die Komponenten durch die \IT{Request for Comments} 4250 bis 4254 der IETF (Internet Engineering Task Force) \cite{ietf}. Ziel der Entwicklung von SSH-2 war es, die Sicherheitsmängel von SSH-1 zu beheben. Außerdem sind weitere Funktionalitäten hinzugekommen. --> evtl. Funktionalitäten aufführen


\subsection{Implementierungen}

Als erste Implementierung des SSH-Protokolls gilt das von Tatu Ylönen 1995 veröffentlichte SSH1.
Es wurde als freie Software veröffentlicht \cite[S. 11]{SSH}. Die Implementierung des SSH-2-Protokolls war hingegen ein kommerzielles Produkt. Die kostenlose Nutzung war nur für Bildungseinrichtungen und den nicht-gewerblichen Bereich erlaubt \cite[S. 22]{SSH}. Folglich kam weiterhin vielerorts die Implementierung des SSH-1-Protokolls zum Einsatz. 1999 entstand das OpenSSH-Projekt, das auf die letzte frei verfügbare Version von SSH aufbaut \cite{OpenSSH}.

Aktuell gibt es verschiedene Implementierungen des SSH-Protokolls. Je nach Einsatzzweck (mobiles Gerät/Workstation) oder Betriebssystem gibt es verschiedene Alternativen. Im Folgenden ist eine Auswahl von Implementierungen mit Kurzbeschreibung aufgeführt.

\subparagraph{Dropbear} ist eine Implementierung des SSH2-Protokolls und steht unter der MIT-Lizenz \cite{dropbear}. Dropbear ist vorallem für Systeme mit geringer Prozessorleistung und geringer Speicherkapazität interessant. Es ist auf verschiedenen POSIX-basierten Systemen lauffähig. Eine entsprechende Auflistung ist auf der offiziellen Webseite von Dropbear \footnote{https://matt.ucc.asn.au/dropbear/dropbear.html\#platforms} zu finden.

\subparagraph{Mosh} (mobile shell) bietet erweiterte Funktionalitäten, vorallem für mobile Geräte. Beispielsweise wird die Verbindung bei Roaming aufrechterhalten. Mosh ist keine direkte Implementierung des SSH-Protokolls. Es überträgt keinen Byte-String zwischen Client und Server. Vielmehr tauschen Client und Server einen Snapshot des aktuellen Bildschirms aus \cite{mosh}.

\subparagraph{Lsh}
\subparagraph{PuTTY} MIT-Lizenz, überwiegend für Windows.
\subparagraph{OpenSSH}




\section{OpenSSH}

OpenSSH -- eine der bekanntesten Implemetierungen des SSH-Protokolls -- dient nun dazu,
elemnentare Funktionalitäten vorzustellen. 

--> beinhaltet Server und Client
--> einfaches Installieren über den Paketmanager

\subsection{Remote Terminal Session}

\begin{program}
[sue@kaktus]\$ ssh micra@login.rz.hs-augsburg.de
micra@login.rz.hs-augsburg.de's password:
%Linux bug 3.2.0-4-amd64 #1 SMP Debian 3.2.65-1+deb7u1 x86_64
Plan your installation, and FAI installs your plan.

Last login: Mon Apr 25 22:38:45 2016
from p5088ff5b.dip0.t-ipconnect.de
micra@bug:~\$

\end{program}

\subsection{Datenübertragung mit \IT{scp}}

\begin{program}
[sue@kaktus ~]\$ scp hello\_all.txt
micra@login.rz.hs-augsburg.de:~
micra@login.rz.hs-augsburg.de's password:
hello\_all.txt        100\% 6297     6.2KB/s   00:00
\end{program}

\subsection{Dateisystem einhängen mit \IT{sshfs}}

\begin{program}
sshfs [user@]host:[dir] mountpoint [options]
\end{program}

\subsection{Public-Key-Authentifizierung}
\subsection{Port-Forwarding}
\subsection{X-Forwarding}
\subsection{Agenten?}


\section{Sicherheitsrelevante Betrachtung}
\subsection{Verschlüsselungsalgorithmen bei SSH}
\subsection{Ablauf einer sicheren Verbindung}\label{sec:ablauf}
\subsection{Aktuelle Schwachstellen}
\subsection{Anforderungen an den Serveradministrator}


\newpage


\section{Zusammenfassung}


\bibliography{sample}
\bibliographystyle{abbrv}


\end{document}


